\documentclass[12pt, a4paper]{article}
\makeatletter
\pagestyle{empty} % すべてのページ番号を消去

\def\id#1{\def\@id{#1}}
\def\department#1{\def\@department{#1}}
\def\@maketitle{
\begin{center}
  % \vspace{10mm}
  {\Large\bf \@title \par}% 論文のタイトル部分
  % \vspace{10mm}
  % {\Large \@date\par} % 提出年月日部分
  % \vspace{3mm}
  % { \@department 学籍番号 \@id  \@author}  % 所属部分
  % {\Large 学籍番号 \@id \par}  % 学籍番号部分
  % \vspace{10mm}
  % {\large \@author}% 氏名 
\end{center}
\par\vskip 1.5em
}
\usepackage[top=20truemm,bottom=30truemm,left=15truemm,right=15truemm]{geometry}
\makeatother
\title{}
% \date{\today}
% \department{ }
% \id{81719437}
% \author{加藤広野}

\begin{document}
\maketitle % ベージ番号を表示するコマンドを呼ぶため,ページ消す場合は下をコメントアウト
\thispagestyle{empty} % ← ココに\thispagestyleを書く

\begin{flushleft}
Hiroya Kato\\
Department of Information and Computer Science, \\
Faculty of Science and Technology, \\
Keio University\\
223-8522\\
Email: kato@sasase.ics.keio.ac.jp 
\end{flushleft}

\begin{flushleft}
Prof. Derek Abbott  \\
Editor-in-Chief \\
\emph{IEEE Access}
\end{flushleft}

\begin{flushleft}
August 24, 2021 
\end{flushleft}

\begin{flushleft}
Dear Prof. Abbott:  \\
\end{flushleft}

\noindent
I would like to submit the manuscript entitled ``Android Malware Detection Based on Composition Ratio of Permission Pairs'' by Hiroya Kato, Takahiro Sasaki, and Iwao Sasase to be considered for publication as a ``Regular'' Article in the \textit{IEEE Access}.  

\vspace{\baselineskip}
\noindent
This study presents a more practical malware detection scheme in the Android platform, which is the most popular and occupies 85\% of the smartphone market share in the world.
Practical detection is imperative because approximately 365,000 new Android malware samples have been found in the third quarter of 2019.
Among various detection schemes, permission pair based ones are promising for practical detection.
However, conventional schemes cannot simultaneously meet requirements for practical use in terms of efficiency, intelligibility, and stability of detection performance. 
Even the latest scheme cannot meet the stability whereas other requirements are met.
This is because recent malware tends to require unnecessary permissions to imitate benign apps, which makes detection difficult.
These facts motivate us to work on this study.
Our scheme is suitable for practical use because all the requirements can be met.
By using real datasets, our results show that our scheme can detect malware with up to 97.3\% accuracy. Besides, compared with a existing scheme, our scheme can reduce the feature dimensions by about 99\% with maintaining comparable accuracy on recent datasets.
We would like to present our results as soon as possible.
Therefore, we determined to submit this work to \textit{IEEE Access}.
This work would contribute to future studies that enhances security on the Android platform.
We believe this study will be of interest to the readers of your journal.

\vspace{\baselineskip}
\noindent
We declare that this manuscript is original, has not been published before and is not currently under construction by another journal.

\vspace{\baselineskip}
\noindent
We state that we have had no previous contact with your journal regarding this submission.
The authors have no conflicts of interest to declare.
As corresponding author, I confirm that the manuscript has been read and approved for submission by all the authors.  

\vspace{\baselineskip}
\noindent
We hope you find our manuscript suitable for publication and look forward to hearing from you in due course.

\begin{flushleft}
Sincerely, \\
Hiroya Kato\\
\end{flushleft}
\end{document}
